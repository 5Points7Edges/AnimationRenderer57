\documentclass[preview]{standalone}

\usepackage[UTF8]{ctex}
\usepackage[english]{babel}
\usepackage{amsmath}
\usepackage{amssymb}
\usepackage{dsfont}
\usepackage{setspace}
\usepackage{tipa}
\usepackage{relsize}
\usepackage{textcomp}
\usepackage{mathrsfs}
\usepackage{calligra}
\usepackage{wasysym}
\usepackage{ragged2e}
\usepackage{physics}
\usepackage{xcolor}
\usepackage{microtype}
\linespread{1}

\setCJKfamilyfont{kaiti}{KaiTi}
\newfontfamily\arial{Arial}
\newfontfamily\courier{Courier New}
\newfontfamily\comic{Comic Sans MS}
\xeCJKsetup{CJKecglue={}}
% pseudocode configuration
\usepackage{algpseudocode, setspace}% http://ctan.org/pkg/algorithmicx
\algtext*{EndWhile}% Remove "end while" text
\algtext*{EndFor}% Remove "end for" text
\algtext*{EndIf}% Remove "end if" text
\algtext*{EndFunction}% Remove "end function" text
\renewcommand\algorithmicthen{} % Remove then
\renewcommand\algorithmicdo{} % Remove do
\usepackage{siunitx}
\usepackage{pifont}
\usepackage{enumitem}


\begin{document}

{\CJKfamily{noto}
    臣亮言:先帝创业未半,而中道崩殂;今天下三分,益州疲敝,此诚
危急存亡之秋也。然侍卫之臣,不懈于内;忠志之士,忘身于外者:盖追
先帝之殊遇,欲报之于陛下也。诚宜开张圣听,以光先帝遗德,恢弘志士
之气;不宜妄自菲薄,引喻失义,以塞忠谏之路也。宫中府中,俱为一体;
陟罚臧否,不宜异同:若有作奸犯科,及为忠善者,宜付有司,论其刑赏,
以昭陛下平明之治;不宜偏私,使内外异法也。侍中、侍郎郭攸之、费依、
董允等,此皆良实,志虑忠纯,是以先帝简拔以遗陛下:愚以为宫中之事,
事无大小,悉以咨之,然后施行,必得裨补阙漏,有所广益。将军向宠,
性行淑均,晓畅军事,试用之于昔日,先帝称之曰“能”,是以众议举宠
为督:愚以为营中之事,事无大小,悉以咨之,必能使行阵和穆,优劣得
所也。亲贤臣,远小人,此先汉所以兴隆也;亲小人,远贤臣,此后汉所
以倾颓也。先帝在时,每与臣论此事,未尝不叹息痛恨于桓、灵也!侍中、
尚书、长史、参军,此悉贞亮死节之臣也,愿陛下亲之、信之,则汉室之
隆,可计日而待也。

    臣本布衣,躬耕南阳,苟全性命于乱世,不求闻达于诸侯。先帝不以
臣卑鄙,猥自枉屈,三顾臣于草庐之中,谘臣以当世之事,由是感激,遂
许先帝以驱驰。后值倾覆,受任于败军之际,奉命于危难之间:尔来二十
有一年矣。先帝知臣谨慎,故临崩寄臣以大事也。受命以来,夙夜忧虑,
恐付托不效,以伤先帝之明;故五月渡泸,深入不毛。今南方已定,甲兵
已足,当奖帅三军,北定中原,庶竭驽钝,攘除奸凶,兴复汉室,还于旧
都:此臣所以报先帝而忠陛下之职分也。至于斟酌损益,进尽忠言,则攸
之、依、允等之任也。愿陛下托臣以讨贼兴复之效,不效则治臣之罪,以
告先帝之灵;若无兴复之言,则责攸之、依、允等之咎,以彰其慢。陛下
亦宜自谋,以谘诹善道,察纳雅言,深追先帝遗诏。臣不胜受恩感激!今
当远离,临表涕泣,不知所云。}

\end{document}
